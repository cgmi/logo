\documentclass{article}
\usepackage[pdftex,active,tightpage]{preview}
%\setlength\PreviewBorder{2mm} % use to add a border around the image
\usepackage{tikz}

% Colors
\definecolor{darkblue}{RGB}{15,155,220}
\definecolor{mediumblue}{RGB}{96,186,227}
\definecolor{lightblue}{RGB}{156,208,232}

% Rectangles
\newcommand{\rectOne}{(0,0),(1,0),(1,1),(2,1),(2,2),(3,2),(4,2),(5,2),(6,2),(1,4)
                     ,(2,5),(4,4),(5,4),(5,3),(6,3),(7,3),(8,3),(9,3),(9,4),(9,2)
                     ,(9,5),(10,5),(8,5),(9,6),(8,6),(7,6),(6,6),(7,7),(6,7),(5,7)
                     ,(4,7),(4,8),(3,8),(10,2),(11,2),(22,-2),(23,-2),(20,1),(20,2)
                     ,(19,3),(18,3),(18,4),(21,1),(22,1),(23,1),(24,1),(25,1)
                     ,(26,1),(14,0),(15,0),(17,-2),(18,-2),(19,-2),(19,-3)}
\newcommand{\rectTwo}{(2,3),(10,3),(16,3),(16,1),(18,1),(16,-1),(18,-1),(20,-1)
                     ,(22,-1),(24,-1),(26,-1),(20,-3)}
\newcommand{\rectThree}{(12,1)}

\begin{document}
\begin{preview}
  \begin{tikzpicture}[scale=0.2]
    \foreach \c in \rectOne {\fill[darkblue] \c rectangle ([shift={(1,1)}] \c;);}
      \foreach \c in \rectTwo { \fill[mediumblue] \c rectangle ([shift={(2,2)}] \c;); }
      \foreach \c in \rectThree { \fill[lightblue] \c rectangle ([shift={(4,4)}] \c;); }
    \node[gray] at (5.5,-2.5) {\Huge{\textsf{cgmi.uni.kn}}};
  \end{tikzpicture}
\end{preview}
\end{document}
